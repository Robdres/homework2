\documentclass{assignment}

\usepackage[utf8]{inputenc}
\usepackage[T1]{fontenc}
\usepackage{babel}
\usepackage{listings}
\usepackage{assignment}
\usepackage{tikz}
\usepackage{pgfplots}
\pgfplotsset{compat=1.18}


\newcommand*{\name}{Roberto Alvarado}
\newcommand*{\id}{00206411}
\newcommand*{\course}{Computer Networks}

\newcommand*{\assignment}{Homework 2}

\begin{document}
\assignmentTitle{\name}{\id}{logo.png}{\course}{\assignment}
\begin{ex}
 Let A and B be two stations attempting to transmit on an Ethernet. Each has a steady queue of
frames ready to send; A’s frames will be numbered A1, A2, and so on, and B’s similarly. Let T=
51.2µs be the exponential backoff base unit. Suppose A and B simultaneously attempt to send
frame 1, collide, and happen to choose backoff times of 0 x T and 1 x T, respectively, meaning
A wins the race and transmits A1 while B waits. At the end of this transmission, B will attempt
to retransmit B1 while A will attempt to transmit A2. These first attempts will collide, but now A
backs off for either 0 x T or 1 x T, while B backs off for time equal to one of 0 x T, . . ., 3 x T. 
\begin{itemize}
  \item Give the probability that A wins this second backoff race immediately after this first collision;
that is A’s first choice of backoff time k x 51.2 is less that B’s.
  \item Suppose A wins this second backoff race. A transmits A3, and when it is finished, A and B
collide again as A tries to transmit A4 and B tries once more to transmit B1. Give the
probability that A wins this third backoff race immediately after the first collision.
  \item Give a reasonable lower bound for the probability that A wins all the remaining backoff
races.
  \item What then happens to the frame B1?
\end{itemize}
This scenario is known as the Ethernet capture effect.
\end{ex}
\begin{itemize}
  \item The idea is that A has to be smaller than B, thus as $k(A) = [0,1]$ and
    $k(B)=[0,1,2,3]$, then the case is that A is 0 and B is any other than 0 or
    A is 1 and B is either 2 or 3, thus, as we expect same probability of each
    case then
    $$P[E=A] = P[A=0,B\neq0] + P[A=1,B>1]$$
    $$P[E=A] = \frac{1}{2}*\frac{3}{4}+ \frac{1}{2}*\frac{2}{4}  $$
    $$P[E=A] = \frac{3}{8} + \frac{2}{8}  $$
    $$P[E=A] = \frac{5}{8} $$
  \item Same idea in this case $k(A) =[0,1]$ and $k(B) = [0,1,2,3,4,5,6,7]$
    $$P[E=A] = P[A=0,B\neq0] + P[A=1,B>1]$$
    $$P[E=A] = \frac{1}{2}*\frac{5}{8}+ \frac{1}{2}*\frac{6}{4}  $$
    $$P[E=A] = \frac{7}{16} + \frac{6}{16}  $$
    $$P[E=A] = \frac{13}{16} $$
\end{itemize}
\end{document}
